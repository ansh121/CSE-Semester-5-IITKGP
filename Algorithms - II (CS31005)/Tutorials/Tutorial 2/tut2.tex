\documentclass{article}
\title{Tutorial 2}
\date{04-08-2019}
\author{Anshul Choudhary(17CS10005)}

\usepackage{amsmath}
\usepackage[ruled,linesnumbered]{algorithm2e}
\usepackage{fixltx2e}

\begin{document}
  \maketitle
  
  \section{Problem Statement}
    
P[1..n] is an input list of n points on xy-plane. Assume that all n points have distinct x-coordinates and
distinct y-coordinates. Let p L and p R denote the leftmost and the rightmost points of P , respectively.
The task is to find the polygon Q with P as its vertex set such that the following conditions are
satisfied.\\\\
i) The upper vertex chain of Q is x-monotone (increasing) from pL to pR.\\
ii) The lower vertex chain of Q is x-monotone (decreasing) from pR to pL.\\
iii) Perimeter of Q is minimum.\\\\
You have to answer the following. Provide necessary figures/diagrams for explanations.\\\\
1. Develop the recurrences needed for DP, with clear arguments.\\
2. Design the algorithm and write its main steps.\\
3. Derive the time and space complexities of your algorithm.\\
  
  \section{Recurrences}
  
  	Let p1,p2, . . . ,pn denote the points by ordered by increasing x-coordinates, l(i,j) denote the length of a shortest path with endpoints p\textsubscript{i} and p\textsubscript{j} , i $<$ j and dist(p\textsubscript{i},p\textsubscript{j}) denote the distance between point pi and pj.

    
      \begin{equation*}
        l(i,j)=\begin{cases}
		   dist(p\textsubscript{i},p\textsubscript{j})\ \ \ \ \ \ \ \ \ \ \  \ \ if\ i = 1\ and\ j = 2 \\
			\\
           l(i,j-1)+dist(p\textsubscript{j-1},p\textsubscript{j}) \ \ \ \ \ \ \ if\ i < j-1\\
			\\
           min\textsubscript{$1 \leq k < i$}\ (l(k,i) + dist(p\textsubscript{k},p\textsubscript{j}))\ \ \ \ if\ j > 2\ \ and \ \ i=j-1\\
               	\end{cases}
        \end{equation*}
  
  \section{Algorithm}
    N(i,j) denotes the neighbour of p\textsubscript{j} in a shortest x-monotone path from p\textsubscript{i} to p\textsubscript{j}\\  
    \begin{algorithm}[H]
	n$\leftarrow$ $\vert$p$\vert$\\
	Compute $l(i,j)\ and\ N(i,j)$, for all $1 \leq i < j < n$.\\
	\For{ j = 2 to n }
	{
		\For{ i = 1 to j-1 }
		{
			\If{ i = 1 and j = 2}
			{
				$l[i,j] \longleftarrow dist(p[i], p[j])$\\
				$N[i,j] \longleftarrow i$\\
			}
			\ElseIf{ $j>i+1$ }
			{
				$l[i,j] \longleftarrow p[i,j-1] + dist(p[j-1],p[j])$\\
				$N[i,j] \longleftarrow j-1$\\
			}
			\Else
			{
				$l[i,j] \longleftarrow +\infty $\\
				\For{ k = 1 to i-1 }
				{
					$q \longleftarrow l[k,i] + dist(p[k],p[j])$\\
					\If{ $q < l[i,j]$ }
					{
						$l[i,j] \longleftarrow q$\\
						$N[i,j] \longleftarrow k$\\
					}	
				}
			}
		}
	}	
     
Construct the path. Stacks S[1] and S[2] will be used to construct the two x-monotone parts of the path.\\
Let S be an array of two initially empty stacks S[1] and S[2].\\

	$k \longleftarrow 1 $\\
	$i=n-1 $\\
	$j=n $\\
	\While{$j>1$}
	{
		PUSH(S[k],j)\\
		$j \longleftarrow N[i,j]$\\
		\If{$j<i$}
		{
			swap $i \longleftrightarrow j$\\
			$k \longleftarrow 3-k$\\
		}
	}

	PUSH(S[1],1)\\
	\While{ S[2] is not empty }
	{
		PUSH(S[1],POP(S[2]))\\
	}
	\For{i=1 to n}{
		$T[i] \longleftarrow POP(S[1])$\\
	}
     \end{algorithm}
     
Array T contains the points in the order they are connected consecutvely.\\
  
  \section{Observations}
  1. Points p\textsubscript{n-1} and p\textsubscript{n} are neighbours in any bitonic tour that visits points $p1, p2, . . . , pn$.\\
  2. Given a normal bitonic path P with endpoints p\textsubscript{i} and p\textsubscript{j} , $i < j$, let p\textsubscript{k} be the neighbour of p\textsubscript{j} on this path. Then the path P' obtained by removing p\textsubscript{j} from P is a normal bitonic path with endpoints p\textsubscript{i} and p\textsubscript{k}. In particular, p\textsubscript{j-1} $∈$ {p\textsubscript{i}, p\textsubscript{k} }. If P is a shortest normal bitonic path with endpoints p\textsubscript{i} and p\textsubscript{j} , then P' is a shortest normal bitonic path with endpoints p\textsubscript{i} and p\textsubscript{k}.\\
  3. Consider the neighbour p\textsubscript{k} of p\textsubscript{j} in a normal bitonic path P with endpoints p\textsubscript{i} and p\textsubscript{j} , $i < j$. If $i=j-1$, we have$ 1 \leq k < i$. If $i < j - 1$, we have $k = j-1$.\\

   
   
  \section{Time and space complexities}
  
  It is easy to see that lines 25-43 take linear time. Indeed, the loop in Lines 42-44 is executed n times and performs constant work per iteration. The loop in Lines 39-40 is executed $\vert S[2]\vert $ times, and each iteration takes constant time; hence, it suffices to prove that $ \vert S[2]\vert \leq n $ after the execution of Lines 25-38. To prove this, it suffices to show that the loop in Lines 30-37 is executed at most n-1 times because every iteration pushes only one entry onto stack S[1] or S[2]. The loop in Lines 30-37 takes constant time per iteration. It is executed until j=1. However, initially j=n, and the computation performed inside the loop guarantees that j decreases by one in each iteration. Hence, the loop is executed at most n-1 times. To analyze the running time of Lines 1-24, we first observe that this part of the algorithm consists of two nested loops; the code in Lines 5-22 is executed $O(n\textsuperscript{2} )$ times because i runs from 2 to n and in every iteration of the outer loop, j runs from 1 to i-1. Now, we perform constant work inside the loop unless i=j-1. In the latter case, we execute the loop in Lines 15-21, (i-1)=O(n) times. Since there are only n-1 pairs (i,j) such that
$1\leq i=j-1 < n-1$, we spend linear time in only n-1 iterations of the two outer loops and constant time in all other iterations. Hence, the running time of this algorithm is,
  \begin{equation*}
 	 Time Complexity=O(n\textsuperscript{2}\ .\ 1 + n\ .\ n) = O(n\textsuperscript{2} ).	 
  \end{equation*}
Since we are using a 2D matrix of size N x N (N $\Longrightarrow no. of points$) in the first part. Rest of the extra space used is of the size of O(n). Hence the Space Complexity will be,
  \begin{equation*}
 	 Space Complexity=O(n\textsuperscript{2})	 
  \end{equation*}
   

\end{document}
\grid
