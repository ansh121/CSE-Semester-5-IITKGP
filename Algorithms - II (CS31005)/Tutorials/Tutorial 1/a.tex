\documentclass{article}
\title{Tutorial 1}
\date{17-07-2019}
\author{Anshul Choudhary(17CS10005)}

\usepackage{amsmath}
\usepackage{algorithm2e}

\begin{document}
  \maketitle
  
  \section{Problem Statement}
    $A[1..m]$ and $B[1..n]$ are two 1D arrays containing $m$ and $n$ integers 
    respectively, where $m\le n$.
    We need to construct a sub-array $C[1..m]$ of $B$ such that 
    $\sum\limits_{i=1}^{m} \big|A[i]-C[i]\big|$ is minimized.
  
  \section{Recurrences}
  
    Lets denote M[i][j] as the minimum value of   $\sum\limits_{j=1}^{i} \big|A[j]-C[j]\big|$ when array $A[1..i]$ and $B[1..j]$ are considered where $i >= j$, $i <= m$ and $j <= n$.
    
      \begin{equation*}
        M[i][j]=\begin{cases}
		   |A[i] - B[j]|\ \ \ \ \ if\ i = 1\ and\ j = 1 \\
			\\
                   min\{ |A[i] - B[j] |  ,  M[i][j-1] \};\ \ \ \ \ if\ i = 1\ and\ j != 1\\
			\\
                   INT\_MAX;\ \ \ \ if\ i > j\\
			\\
                   min\{ |A[i] - B[j] | + M[i-1][j-1]  ,  M[i][j-1] \}\ \ \ \ otherwise \\
                \end{cases}
        \end{equation*}

Here M[n][m] is the final ans.
  
  \section{Algorithm}
  
    \begin{algorithm}[H]
	int M[n][m]\\
	\For{ i = 1 to n }
	{
		\For{ j = 1 to m }
		{
			\If{ i == 1}
			{
				\If{j==1}
				{
					M[i][j] = ${|A[i] - B[j]}$ \;
				}
				\Else
				{
					M[i][j] = ${min\{|A[i] - B[j],\ M[i][j-1]\}}$ \;
				}
			}
			\ElseIf{ i $>$ j }
			{
				M[i][j] = INT\_MAX \;
			}
			\Else
			{
				M[i][j] = ${min\{|A[i] - B[j]| + M[i-1][j-1],\ M[i][j-1]\}}$ \;
			}
		}
	}
	
	
	
	char B[m]\\
	i=m\\
	j=n\\
	
	while(j!=0)\\
	\{\\
		\ \ \ while($i>0 ans M[j][i]==M[j][i-1]$)\\
	\ \ \ 	\{\\
		\ \ \ \ 	i--\\
		\ \ \ \}\\
		\ \ \ $C[m-i]=B[j][i]$\\
		\ \ \ j--\\
	\}\\
	
	

     \end{algorithm}
  
  \section{Demonstration}
  
  Lets take an example ,
  \begin{equation*}
  	A=[4\ 5\ 8\ 6\ 7]\\
  \end{equation*}
  \begin{equation*}
 	 B=[2\ 4\ 3\ 1]
  \end{equation*}
  So the matrix M created is,
   
  \begin{equation*}
 	 M=\left[\begin{matrix}
 	 	2 & 2 & 2 & 2 & 2\\
 	 	X & 3 & 3 & 3 & 3\\
 	 	X & X & 8 & 6 & 6\\
 	 	X & X & X & 13 & 12
 	 \end{matrix}\right]
  \end{equation*}
  
By calculating C[] from above matrix, we get,
  \begin{equation*}
 	 C=[4\ 5\ 6\ 7]
  \end{equation*}
   
   
  \section{Time and space complexities}
  
  From the above pseudo code we can calculate Time Complexity as,\\
  	(i) outer for loop will run n times ----$>$ O(n)\\
  	(ii) inner for loop will run m times ----$>$ O(m)\\
  So total time will be,
  \begin{equation*}
 	 T=O(n)*O(m)	 
  \end{equation*}
  \begin{equation*}
 	 Time Complexity=O(n*m)	 
  \end{equation*}
Since we are using a 2D matrix of size m x n, the Space Complexity will be,\\
  \begin{equation*}
 	 Space Complexity=O(n*m)	 
  \end{equation*}
   

\end{document}
\grid
